% template presente
\documentclass[a4paper,12pt]{article}

\usepackage{tabularx}
\usepackage{tcolorbox}
\usepackage[utf8]{inputenc}
\usepackage[italian]{babel}
\usepackage{hyperref}
\hypersetup{
	colorlinks=true,
	linkcolor=blue,
	filecolor=blue,
	urlcolor=blue,     
}
\usepackage{graphicx}
\graphicspath{{resources/}{../resources/}{../../resources/}{../../../resources/}{../../../../resources/}}
\usepackage{xcolor}
\usepackage{geometry}
\usepackage{setspace}
\usepackage{colortbl}
\usepackage{hyperref} 
\usepackage{fancyhdr} 
\usepackage{titlesec}
\geometry{margin=2.5cm}
% ===== Variabile per il titolo del documento =====
\newcommand{\documenttitle}{Verbale Riunione Esterna Numero 1}


\setlength{\parindent}{0pt}
\setstretch{1.2}

% ===== Stile intestazione =====
\pagestyle{fancy}
\fancyhf{}
\fancyhead[L]{\textcolor{gray}{\documenttitle - BitByBit}}
\fancyfoot[C]{\thepage}

% ===== Formattazione sezioni =====
\titleformat{\section}{\Large\bfseries}{\thesection.}{0.5em}{}
\titleformat{\subsection}{\large\bfseries}{\thesubsection.}{0.5em}{}

\begin{document}
	
	% ======= HEADER UNIVERSITÀ E GRUPPO CENTRATI VERTICALMENTE =======
	\vspace*{\fill} % --- Spinge verso il basso l'inizio del contenuto
	
	\begin{center}
		\begin{minipage}{0.25\textwidth}
			\centering
			\includegraphics[width=\linewidth]{logoUni.png}
		\end{minipage}
		\hfill
		\begin{minipage}{0.7\textwidth}
			\raggedright
			{\color{red}\LARGE \textbf{Università degli Studi di Padova}}\\[0.3cm]
			{\large
				Laurea: Informatica\\
				Corso: Ingegneria del Software\\
				Anno Accademico: 2025/26
			}
		\end{minipage}
	\end{center}
	
	\vspace{1cm}
	
	\begin{center}
		\begin{minipage}{0.25\textwidth}
			\centering
			\includegraphics[width=\linewidth]{logo.png}
		\end{minipage}
		\hfill
		\begin{minipage}{0.7\textwidth}
			\raggedright
			{\LARGE \textbf{Gruppo 17}}\\[0.3cm]
			{\large
				Nome: BitByBit\\
				Email: swe.bitbybit@gmail.com
			}
		\end{minipage}
	\end{center}
	
	\vspace{1.5cm}
	
	\begin{center}
		{\LARGE \textbf{\documenttitle}}
	\end{center}
	
	\vspace*{\fill} % --- Spinge verso l’alto la fine del contenuto, centrando tutto il blocco
	
	\clearpage
	% ======= INFO GENERALI =======
	\section*{Informazioni Generali}
	\renewcommand{\arraystretch}{1.3}
	
	\begin{tcolorbox}
		\begin{tabularx}{\textwidth}{@{}l X@{}}
			\textbf{Redattore:} & Sanguin Marco \\
			\textbf{Verificatore:} & Parolin Dennis \\
			\textbf{Data:} & 18 Novembre 2025 \\
			\textbf{Durata:} & 45m \\
			\textbf{Luogo:} & Google Meet \\
		\end{tabularx}
	\end{tcolorbox}
	
	\vspace{0.4cm}
	\textbf{Partecipanti:}
	\begin{itemize}
		\item Gruppo BitByBit
		\begin{itemize}
			\item Manisi Riccardo
			\item Parolin Dennis
			\item Scaggiante Gabriele
			\item Sanguin Marco
			\item Visentin Giovanni
			\item Fracasso Ferdinando
		\end{itemize}
		
		\item Miriade
		\begin{itemize}
			\item Arianna Bellino
			\item Emanuele Righetto
			\item Annalisa Egidi
		\end{itemize}
	\end{itemize}
	
	\textbf{Assenti:}
	\begin{itemize}
		\item (nessuno)
	\end{itemize}
	
	\clearpage
	
	\vspace{0.5cm}
	
	\vspace{0.8cm}
	
	% ======= INDICE SU PAGINA DEDICATA =======
	\clearpage
	\tableofcontents
	\thispagestyle{empty} % senza numero di pagina per l'indice
	\clearpage
	
	% ---------- ORDINE DEL GIORNO ----------
	\section{Ordine del Giorno}
	\begin{itemize}
		\item Presentare le idee del gruppo riguardo le funzionalit\`a dell'applicazione da sviluppare
		\item Chiarire alcuni dubbi sorti durante la scorsa riunione interna riguardo le funzionalit\`a
		\item Allinearsi su come procedere per iniziare il primo sprint
	\end{itemize}
	
	% ---------- DISCUSSIONI ----------
	\section{Discussioni}
	La riunione \`e iniziata con la presentazione delle funzionalit\`a che il gruppo BitByBit proponeva di poter sviluppare, in particolare riguardo:
	\begin{itemize}
		\item la struttura generale dell'applicazione
		\item il coinvolgimento e utilizzo dell'intelligenza artificiale come chatbot
		\item la sezione del diario segreto
		\item i sistemi di allarme e notifica a contatti fidati
		\item mappa interattiva di luoghi sicuri
		\item analisi passiva dell'utilizzo dell'applicazione e raccolta di dati statistici
	\end{itemize}
	Lo scopo della presentuazione era ricevere un riscontro riguardo le funzionalit\`a proposte in relazione alle aspettative della proponente e avere una stima dei tempi e di fattibilit\`a, considerando anche la loro numerosit\`a.
	
	In risposta, i responsabili di Miriade hanno consigliato di fare un passo indietro e procedere con un incontro in presenza per discutere delle funzionalit\'a e del design, anche con l'aiuto di una loro collega pi\`u esperta: ha quindi invitato il gruppo marted\`i  25 novembre nella sede Miriade di Padova per iniziare con il desing sprint e in particolare parlare delle funzionalit\`a principali dell'applicazione, trovare i suoi punti di forza e quelli che potrebbe essere di debolezza, fare una prima analisi dell'interfaccia utente e un mock up iniziale.
	
	La riunione si \`e conclusa con il chiarimento di alcuni dubbi riguardo la redazione dei documenti interni e sul canale asincorono da utilizzare per una comunicazione pi\`u efficente, concordando sull'utilizzo di Google Chat, allineandosi con il metodo dell'azienda Miriade.
	
	
	% ---------- DECISIONI ----------
	\section{Decisioni}
	\begin{center}
		\small
		\renewcommand{\arraystretch}{1.2} 
		\arrayrulecolor{black} 
		\begin{tabular}{|p{0.73\textwidth}|c|}
			\hline
			\rowcolor{gray!60} 
			\textbf{Descrizione decisione} & \textbf{Codice decisione} \\
			\hline
			\rowcolor{white}
			Fissato incontro in presenza Marted\`i 25 novembre & VI C 1.1 \\
			\hline
			\rowcolor{gray!20}
			Utilizzo di Google Chat come canale asincrono di comunicazione tra gruppo e proponente &  VI C 1.2 \\
			\hline
		\end{tabular}
	\end{center}
	
	% ---------- TO DO ----------
	\section{To Do}
	Dalle discussioni e decisioni intraprese, non sono emerse attivit\`a da fare ma ognuno si \`e preso il compito di analizzare nuovamente le funzionalit\`a dell'applicazione per arrivare preparati al prossimo incontro.
	
	
	% ---------- REDAZIONE E REVISIONE ----------
	\clearpage
	\section{Redazione e revisioni del documento}
	
	\begin{center}
		\small
		\renewcommand{\arraystretch}{1.2} 
		\arrayrulecolor{black}
		\begin{tabular}{|c|p{0.11\linewidth}|p{0.21\linewidth}|p{0.23\linewidth}|p{0.21\linewidth}|}
			\hline
			\rowcolor{gray!60} 
			\textbf{Versione} & \textbf{Data} & \textbf{Autore} & \textbf{Descrizione} & \textbf{Verificatore} \\
			\hline
			\rowcolor{white}
			0.2.1 & 25-11-19 & Sanguin Marco & Aggiunto spazio per firma & Parolin Dennis\\
			\hline
			\rowcolor{gray!20}
			0.2.0 & 25-11-19 & Sanguin Marco & Modifica e aggiunta informazioni &  Parolin Dennis\\
			\hline
			\rowcolor{white}
			0.1.0 & 25-11-19 & Sanguin Marco & Redazione e creazione file & Parolin Dennis \\
			\hline
			
		\end{tabular}
	\end{center}

	% ---------- FIRMA AZIENDALE ----------
\vspace*{\fill} % Spinge la firma in fondo alla pagina
\noindent
\begin{minipage}{0.60\textwidth}
    {\small
    \textbf{Firma aziendale:}\\[0.3cm]
    \textit{(Spazio riservato all’azienda per apporre firma o timbro)}\\[0.8cm]
    \fbox{\rule{0pt}{2.5cm}\rule{5cm}{0pt}} % rettangolo segnaposto firma
    }
\end{minipage}
	
\end{document}
