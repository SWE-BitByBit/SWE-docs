% template presente
\documentclass[a4paper,12pt]{article}

\usepackage{tabularx}
\usepackage{tcolorbox}
\usepackage[utf8]{inputenc}
\usepackage[italian]{babel}
\usepackage{hyperref}
\hypersetup{
    colorlinks=true,
    linkcolor=blue,
    filecolor=blue,
    urlcolor=blue,     
}
\usepackage{graphicx}
\graphicspath{{resources/}{../resources/}{../../resources/}{../../../resources/}{../../../../resources/}}
\usepackage{xcolor}
\usepackage{geometry}
\usepackage{setspace}
\usepackage{colortbl}
\usepackage{hyperref} 
\usepackage{fancyhdr} 
\usepackage{titlesec}
\geometry{margin=2.5cm}
% ===== Variabile per il titolo del documento =====
\newcommand{\documenttitle}{Verbale Riunione Numero 2}


\setlength{\parindent}{0pt}
\setstretch{1.2}

% ===== Stile intestazione =====
\pagestyle{fancy}
\fancyhf{}
\fancyhead[L]{\textcolor{gray}{\documenttitle - BitByBit}}
\fancyfoot[C]{\thepage}

% ===== Formattazione sezioni =====
\titleformat{\section}{\Large\bfseries}{\thesection.}{0.5em}{}
\titleformat{\subsection}{\large\bfseries}{\thesubsection.}{0.5em}{}

\begin{document}

% ======= HEADER UNIVERSITÀ E GRUPPO CENTRATI VERTICALMENTE =======
\vspace*{\fill} % --- Spinge verso il basso l'inizio del contenuto

\begin{center}
    \begin{minipage}{0.25\textwidth}
        \centering
        \includegraphics[width=\linewidth]{logoUni.png}
    \end{minipage}
    \hfill
    \begin{minipage}{0.7\textwidth}
        \raggedright
        {\color{red}\LARGE \textbf{Università degli Studi di Padova}}\\[0.3cm]
        {\large
        Laurea: Informatica\\
        Corso: Ingegneria del Software\\
        Anno Accademico: 2025/26
        }
    \end{minipage}
\end{center}

\vspace{1cm}

\begin{center}
    \begin{minipage}{0.25\textwidth}
        \centering
        \includegraphics[width=\linewidth]{logo.png}
    \end{minipage}
    \hfill
    \begin{minipage}{0.7\textwidth}
        \raggedright
        {\LARGE \textbf{Gruppo 17}}\\[0.3cm]
        {\large
        Nome: BitByBit\\
        Email: swe.bitbybit@gmail.com
        }
    \end{minipage}
\end{center}

\vspace{1.5cm}

\begin{center}
    {\LARGE \textbf{\documenttitle}}
\end{center}

\vspace*{\fill} % --- Spinge verso l’alto la fine del contenuto, centrando tutto il blocco

\clearpage
% ======= INFO GENERALI =======
\section*{Informazioni Generali}
\renewcommand{\arraystretch}{1.3}

\begin{tcolorbox}
\begin{tabularx}{\textwidth}{@{}l X@{}}
\textbf{Redattore:} & Visentin Giovanni \\
\textbf{Verificatore:} & Parolin Dennis \\
\textbf{Data:} & 17 Novembre 2025 \\
\textbf{Durata:} & 1h 30 min \\
\textbf{Luogo:} & Padova, Conference room \\
\end{tabularx}
\end{tcolorbox}

\vspace{0.4cm}
\textbf{Partecipanti:}
\begin{itemize}
    \item Manisi Riccardo
    \item Parolin Dennis
    \item Scaggiante Gabriele
    \item Sanguin Marco
    \item Visentin Giovanni
    \item Fracasso Ferdinando
\end{itemize}

\textbf{Assenti:}
\begin{itemize}
    \item (nessuno)
\end{itemize}

\clearpage

\vspace{0.5cm}

\vspace{0.8cm}

% ======= INDICE SU PAGINA DEDICATA =======
\clearpage
\tableofcontents
\thispagestyle{empty} % senza numero di pagina per l'indice
\clearpage

% ---------- ORDINE DEL GIORNO ----------
\section{Ordine del Giorno}
\begin{itemize}
    \item Scelta e discussione funzionalità per l'applicazione
    \item Preparazione domande per meeting di domani
    \item Discussione ruoli e ripartizione ore
\end{itemize}

% ---------- DISCUSSIONI ----------
\section{Discussioni}
Durante la riunione abbiamo parlato principalmente delle funzionalità che avremmo voluto implementare all'interno dell'applicazione "L'app che Protegge e Trasforma".
Per trovare un accordo tra tutti i componenti del gruppo abbiamo sottolineato tutte le proposte dell'azienda in rosso (non interessanti/non di prima importanza), in giallo (da chiarire/non di prima importanza) e verde (interessanti/utili).\\
Alcune delle funzionalità che il gruppo ha ritenuto migliori e più interessanti per un'applicazione sono:
\begin{itemize}
    \item La Guida al Coraggio: "Posso Denunciare? Che Aiuti ho?" - Sezione informativa con banca dati geo-localizzata di centri antiviolenza, numeri di emergenza e procedure legali;
    \item Analisi pattern di utilizzo dell'app - Monitoraggio di comportamenti anomali come tentativi di disinstallazione forzata o accessi da luoghi insoliti;
    \item Attivazione alert tramite temporizzatori di sicurezza - Timer preimpostati dall'utente per controlli automatici;
    \item Funzionalità di "Check-in" di sicurezza - Timer periodico che invia alert automatico se l'utente non conferma la propria sicurezza;
    \item Sistema di notifica discreto e personalizzabile - Alert configurabili per contatti di emergenza;
    \item Attivazione alert tramite gesti specifici sullo schermo - Comandi touch discreti per situazioni di emergenza;
    \item Interfaccia per ricerca servizi di supporto geo-localizzati - Strumento intuitivo per trovare aiuto nelle vicinanze;
    \item Integrazione con servizi di chiamata e messaggistica per emergenze - Connessione diretta ai servizi di soccorso;
    \item Messaggi di aiuto rapidi e discreti preimpostati - Template pronti per comunicazioni di emergenza;
    \item Aggiornamento costante banca dati risorse - Sistema di verifica e validazione periodica delle informazioni;
    \item Diario criptato - Registrazione sicura di eventi e pensieri con possibilità di allegare foto o audio protetti;
    \item Strumenti per pianificazione percorsi sicuri - Mappe interattive con rifugi e punti di interesse per emergenze;
    \item Quiz e test di autovalutazione - Strumenti per rafforzare consapevolezza con feedback personalizzati.
\end{itemize}
Oltre alle funzionalità, il gruppo ha pensato e raccolto alcune domande da porre all'azienda Miriade.\\
Inoltre ci siamo anche concentrati sulla questione dei ruoli e della assegnazione delle ore, visti i dubbi di alcuni dei membri del gruppo.

% ---------- DECISIONI ----------
\section{Decisioni}
\begin{center}
\small
\renewcommand{\arraystretch}{1.2} 
\arrayrulecolor{black} 
\begin{tabular}{|p{0.73\textwidth}|c|}
\hline
\rowcolor{gray!60} 
\textbf{Descrizione decisione} & \textbf{Codice decisione} \\
\hline
\rowcolor{white}
Diminuzione ore produttive attese per le prime 2 settimane & VI RTB 2.1 \\
\hline
\rowcolor{gray!20}
Creazione di un insieme di funzionalità base da proporre alla'zienda Miriade &  VI RTB 2.2 \\
\hline
\rowcolor{white}
Preparazione di altre domande generiche da fare all'azienda Miriade & VI RTB 2.3 \\
\hline
\end{tabular}
\end{center}

% ---------- TO DO ----------
\section{To Do}
Dalle discussioni e decisioni intraprese, sono emerse le seguenti attività:

\begin{center}
\small
\renewcommand{\arraystretch}{1.2} 
\arrayrulecolor{black} 
\begin{tabular}{|p{0.52\textwidth}|c|c|}
\hline
\rowcolor{gray!60} 
\textbf{Task} & \textbf{Codice decisione} & \textbf{N°issue GitHub} \\
\hline
\rowcolor{white}
Elaborare presentazione funzionalità per proponente & VI RTB 2.2 & \href{https://github.com/SWE-BitByBit/SWE-project/issues/81}{\textbf{issue 81}} \\
\hline
\rowcolor{gray!20}
Preparare documento domande generiche e gestione del progetto per il meeting con Miriade & VI RTB 2.3 & \href{https://github.com/SWE-BitByBit/SWE-project/issues/80}{\textbf{issue 80}} \\
\hline
\end{tabular}
\end{center}

% ---------- REDAZIONE E REVISIONE ----------
\clearpage
\section{Redazione e revisioni del documento}

\begin{center}
\small
\renewcommand{\arraystretch}{1.2} 
\arrayrulecolor{black}
\begin{tabular}{|c|p{0.11\linewidth}|p{0.21\linewidth}|p{0.23\linewidth}|p{0.21\linewidth}|}
\hline
\rowcolor{gray!60} 
\textbf{Versione} & \textbf{Data} & \textbf{Autore} & \textbf{Descrizione} & \textbf{Verificatore} \\
\hline
\rowcolor{white}
0.2.1 & 25-11-17 & Dennis Parolin & Migliorate le frasi nella sezione Decisioni e To Do. Sistemato il versionamento e l'ordine di presentazione delle righe nelle tabelle  & Riccardo Manisi \\
\hline
\rowcolor{gray!20}
0.2.0 & 25-11-17 & Visentin Giovanni & Modifica e aggiunta informazioni & Parolin Dennis\\
\hline
\rowcolor{white}
0.1.0 & 25-11-17 & Visentin Giovanni & Redazione e creazione file & Parolin Dennis \\
\hline
\end{tabular}
\end{center}

\end{document}